Git (engl. \it{Blödmann}) dient zur Versionsverwaltung von Dateien.
Es zeichnet sich aus durch höchste Geschwindigkeit, simples Design( Konsole?!) und
besonders intuitive und einfache Steuerung, die es uns ermöglicht hat, in Kürze (circa 12 stunden) 
ein nahezu funktionsfähiges Netzwerk aufzusetzen.
Zu allererst muss ein ''Repository'' auf der Seite \url{https://github.com/} erstelllt werden. Freundlicherweise hat sich
Herr Moritz um diese Aufgabe gekümmert.
Anschließend können Daten geladen (''pull''-Funktion) 
