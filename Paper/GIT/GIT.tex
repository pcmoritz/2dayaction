Git (engl. \textit{Blödmann} ) dient zur Versionsverwaltung von Dateien.
Es zeichnet sich aus durch höchste Geschwindigkeit, simples Design( Konsole?!) und
besonders intuitive und einfache Steuerung, die es uns ermöglicht hat, in kürzester Zeit (circa 12 Stunden) 
ein nahezu funktionsfähiges Netzwerk aufzusetzen.
Zu allererst muss ein ''Repository'' auf der Seite \url{https://github.com/} erstelllt werden. 
Freundlicherweise hat sich Herr Moritz um diese Aufgabe gekümmert.
Anschließend können Daten geladen (''pull''-Funktion) und hochgeladen (''push''-Funktion).
Üblicherweise sieht der Uploadvorgang wie folgt aus
\begin{verbatim}
	$user git add .
	$user git commit -m "Kommentar"
	$user git push
\end{verbatim}

Aufgrund der Tatsache, dass alle Beteiligten die Files nach belieben ändern konnten, kam es häufig zu 
folgender Warnung
\begin{verbatim}
	$CONFLICT (content): Merge conflict in abalone.cc.orig
Automatic merge failed; fix conflicts and then commit the result.
\end{verbatim}
In diesem Fall muss ''gemerged'' werden.
\begin{verbatim}
	$user git mergetool -t meld
\end{verbatim}
Oftmals ist das nicht ausreichend und wir empfehlen die Befehle
\begin{verbatim}
	git add . ; git commit -m "Kommentar"; git push;
	git mergetool -t meld; git push
\end{verbatim}
in möglichst zufälliger Reihenfolge auszuführen.
