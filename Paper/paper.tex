%%%%%%%%%%%%%%%%%%%%%%%%%%%%%%%%%%%%%%%%%%%%%%%%%%%%%%%%%%%%%%%%%%%%%%%%
% LaTeX  template by example for your WUPES'12 paper.
%
% Please, keep the recommended length of 8 - 12 pages.
%
% The final formatting will be handled by us.
%%%%%%%%%%%%%%%%%%%%%%%%%%%%%%%%%%%%%%%%%%%%%%%%%%%%%%%%%%%%%%%%%%%%%%%%
\documentclass[a4paper,twoside]{article}

\usepackage[utf8]{inputenc}
\usepackage[english]{babel}
\usepackage[numbers]{natbib}
\usepackage{amsmath, amsthm, amssymb}
\usepackage{url}
\usepackage{paralist}
%\usepackage[small, compact]{titlesec}
\usepackage{graphicx}

\newtheorem{theorem}{Theorem}
\newtheorem{lemma}{Lemma}
\newtheorem{corollary}{Korollar}

\usepackage{wrapfig}

\usepackage{verbatim}

\usepackage[small,bf,labelsep=period]{caption}
\captionsetup{%
  figurename=Fig.,
  tablename=Tab.
}
\usepackage{subfig}

\newcommand{\E}{\operatorname{E}}
\newcommand{\Var}{\operatorname{Var}}

\newcommand{\R}{\operatorname{roots}}
\newcommand{\U}{\operatorname{U}}
\newcommand{\D}{D}
\newcommand{\pa}{\operatorname{pa}}
\newcommand{\de}{\operatorname{de}}
\newcommand{\an}{\operatorname{an}}
\newcommand{\nd}{\operatorname{nd}}
\newcommand{\pperp}{\perp\!\!\!\perp}
\newcommand{\N}{\mathbb{N}}
%\renewcommand{\R}{\mathbb{R}}
\renewcommand{\O}{\mathcal{O}}
\newcommand{\ri}{\operatorname{de}}

% \title{{\sc On partially observed Bayesian networks that allow the
% information-theoretic assertion of common causes}}
\title{{\sc ARES-Abalone}}

\author{
{
\small \bf Philipp Moritz}\\
 \small\texttt{pcmoritz@googlemail.com} 
  \and
 {\small \bf Ulrich Müller} \\ 
 \small \texttt{ulimllr@googlemail.com}
	\and
  {\small \bf Jean-Nicolas Lang}\\
  \small\texttt{jlang@physik.uni-wuerzburg.de}
  \and
  {\small \bf Bijan Chokoufe Néjad}\\
  \small\texttt{bijan.chokoufenejad@physik.uni-wuerzburg.de}
}

\date{}
%%%%%% DEFINICE ZAHLAVI
\newfont{\m}{cmr8}
\newfont{\ms}{cmsl8}
\pagestyle{myheadings}
\markboth{\m P. Moritz, J. Reichardt and N. Ay}
%{\ms On Bayesian networks that allow the
% information-theoretic assertion of common causes}
{\ms A new common cause principle for Bayesian networks}
%%%%%% KONEC DEFINICE ZAHLAVI

\begin{document}

\renewcommand{\bibsection}{\section*{References}}

\maketitle

\begin{abstract}
  Abstract
    
\bigskip

\noindent {\bf Keywords:} Keywords
\bigskip

\end{abstract}
\section{Introduction}
Sinn der 2dayaction war, sich innerhalb von 48 Stunden intensiv mit einem Thema, in unserem 
Fall die Programmierung einer KI für Abalone, zu beschäftigen und dabei möglichst viel zu lernen.
Wir haben neben den Erfahrungen im Programmierung viel über die Arbeitsteilung und 
die Verwaltung von (eigentlich nur sehr großen) Programmierprojekten, die Funktionsweise von SpieleKIs und das Freuermachen
ohne Feuerzeug gelernt.
Innerhalb der zwei Tagen sind mehrere Versionen von KIs entstandanden, 
die zunächst nur zufällige Züge ausführte, in unserer Endversion aber drei Züge
im Voraus mit Alpha-Beta-Suche rechnete und durchaus brauchbaren Ergebnissen lieferte.


\section{Abalone}
Abalone ist ein Brettspiel für zwei Spieler. Jeder Spieler besitzt dazu in der 
Grundstellung 14 Spielsteine auf einem hexagonalen Brett. Abwechselnd sind Züge mit einem, zwei oder drei linear
zusammenhängenden Steinen in die selbe Richtung erlaubt. In der Regel kann ein Zug
von gegnerischen Steinen blockiert werden. Eine Möglichkeit, gegnerische Steine
zu verschibene, stellen die sogenannten Sumitos dar. Dabei werden zwei oder drei 
Kugeln in einer Reihe bewegt und können eine Minderheit gegnerischer Kugeln
wegschieben (also 3 gegen 1, 3 gegen 2, oder 2 gegen 1). Ziel des Spiel ist es 6 Spielsteine
des Gegners vom Brett zu schieben.\\
Abalone wird von der Denksportorganisation gefördert und ist Bestandteil der 
Denk-Sport-Plympiade.


\section{Heuristik}
Heuristiken werden in der Informatik eingesetzt, um den Rechenaufwand zu für
die Lösung eines Problems gering zu halten. Mit dem Einsatz einer Heuristik wird
auf die Garantie einer optimalen Lösung verzichtet, die z.B. beim systematischen 
überprüfen alles Spielabläufe entstehen würde. Dafür wird die Laufzeit des 
Programmes durch verschiedene Faustregeln und Schätzungen bewertet und der Zug 
ausgeführt, der nach wenigen Zügen zur besten Bewertung führt.
Für die Bewertung einer Stellung, kamen folgende Bewertungskriterien
zu Einsatz:
\begin{itemize}
	\item Wie weit sind die Steine in der Mitte des Brettes?
	\item Wie sehr befinden sich die eigenen Steine auf einem Haufen?
	\item Können gegnerische Steine geschlagen werden?
	\item Was für günstige/ungüstige Muster kommen an der Grenze der zwei Spieler vor?
\end{itemize}
Mit diesen Kriterien wurde jeder Stellung eine "Energie" zugeordnet, die wie folgt
berechnet wurde:
\begin{align*}
	H_{ges} &= H_{eigen} + H_{surface} + H_{interface} + V(r)\\
H_{eigen} &=  \lambda_{1} \sum_{(i,j)} \rm{Spielerfarbe}(i,j)\\
H_{surface} &= \lambda_{2} \sum_{(i,j)} \sum_{(p,q)} \left \langle (i,j)\right| \left|  pq\right\rangle  \delta((p,q) \; \; \rm{ist \;leer})\\
H_{interface} &= \lambda_{3} \sum_{\rm{Muster}} \sum_{\rm{Richtungen}} \delta( \text{wenn Muster gefunden}) \\\ 
V(r) &= \lambda_{4}  \sum_{(i,j)}  V(r(i,j))
\end{align*}
Jede hypothetische Stellung, die nach einem möglichen Zug entstehen wird bewertet
und die Züge bevorzugt, die Hamiltonfunktion maximieren.



\section{Algorithmik}
Alpha-Beta-Suche


\section{GIT}
Git (engl. \textit{Blödmann} ) dient zur Versionsverwaltung von Dateien.
Es zeichnet sich aus durch höchste Geschwindigkeit, simples Design( Konsole?!) und
besonders intuitive und einfache Steuerung, die es uns ermöglicht hat, in kürzester Zeit (circa 12 Stunden) 
ein nahezu funktionsfähiges Netzwerk aufzusetzen.
Zu allererst muss ein ''Repository'' auf der Seite \url{https://github.com/} erstelllt werden. 
Freundlicherweise hat sich Herr Moritz um diese Aufgabe gekümmert.
Anschließend können Daten geladen (''pull''-Funktion) und hochgeladen (''push''-Funktion).
Üblicherweise sieht der Uploadvorgang wie folgt aus
\begin{verbatim}
	$user git add .
	$user git commit -m "Kommentar"
	$user git push
\end{verbatim}

Aufgrund der Tatsache, dass alle Beteiligten die Files nach belieben ändern konnten, kam es häufig zu 
folgender Warnung
\begin{verbatim}
	$CONFLICT (content): Merge conflict in abalone.cc.orig
Automatic merge failed; fix conflicts and then commit the result.
\end{verbatim}
In diesem Fall muss ''gemerged'' werden.
\begin{verbatim}
	$user git mergetool -t meld
\end{verbatim}
Oftmals ist das nicht ausreichend und wir empfehlen die Befehle
\begin{verbatim}
	git add . ; git commit -m "Kommentar"; git push;
	git mergetool -t meld; git push
\end{verbatim}
in möglichst zufälliger Reihenfolge auszuführen.


\section{Leben während der 2dayaction}
Innerhalb der zwei Tage haben wir stets versucht uns gesund zu ernähren und alles was wir
zum Überleben brauchten in der Natur oder der nächten Pizzaria zu finden.



\bibliographystyle{dinat}
\bibliography{lit}
\end{document}
