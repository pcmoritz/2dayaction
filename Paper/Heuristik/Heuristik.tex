Heuristiken werden in der Informatik eingesetzt, um den Rechenaufwand zu für
die Lösung eines Problems gering zu halten. Mit dem Einsatz einer Heuristik wird
auf die Garantie einer optimalen Lösung verzichtet, die z.B. beim systematischen 
überprüfen alles Spielabläufe entstehen würde. Dafür wird die Laufzeit des 
Programmes durch verschiedene Faustregeln und Schätzungen bewertet und der Zug 
ausgeführt, der nach wenigen Zügen zur besten Bewertung führt.
Für die Bewertung einer Stellung, kamen folgende Bewertungskriterien
zu Einsatz:
\begin{itemize}
	\item Wie weit sind die Steine in der Mitte des Brettes?
	\item Wie sehr befinden sich die eigenen Steine auf einem Haufen?
	\item Können gegnerische Steine geschlagen werden?
	\item Was für günstige/ungüstige Muster kommen an der Grenze der zwei Spieler vor?
\end{itemize}
Mit diesen Kriterien wurde jeder Stellung eine "Energie" zugeordnet, die wie folgt
berechnet wurde:
\begin{align*}
	H_{ges} &= H_{eigen} + H_{surface} + H_{interface} + V(r)\\
H_{eigen} &=  \lambda_{1} \sum_{(i,j)} \rm{Spielerfarbe}(i,j)\\
H_{surface} &= \lambda_{2} \sum_{(i,j)} \sum_{(p,q)} \left \langle (i,j)\right| \left|  pq\right\rangle  \delta((p,q) \; \; \rm{ist \;leer})\\
H_{interface} &= \lambda_{3} \sum_{\rm{Muster}} \sum_{\rm{Richtungen}} \delta( \text{wenn Muster gefunden}) \\\ 
V(r) &= \lambda_{4}  \sum_{(i,j)}  V(r(i,j))
\end{align*}
Jede hypothetische Stellung, die nach einem möglichen Zug entstehen wird bewertet
und die Züge bevorzugt, die Hamiltonfunktion maximieren.

